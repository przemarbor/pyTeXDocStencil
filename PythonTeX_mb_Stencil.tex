\documentclass[11pt]{article}

%%%%%%%%%%%%%%%%%%%%%%%%%%%%%%%%%%%%%%%%%%%%%%%%%%%%%%%%%%%%%%%%
%%%%%%%%%%%%%%%%%%%%%%%%%%%%%%%%%%%%%%%%%%%%%%%%%%%%%%%%%%%%%%%%
%%%%%%%%%%%%%%%%%%%%%%%%%%%%%%%%%%%%%%%%%%%%%%%%%   pythontex ->

% % % instead of including \usepackage{mbpytex}

% Extracts from the Poore's python_gallery.tex:
% https://github.com/gpoore/pythontex/tree/master/pythontex_gallery


% Engine-specific settings
% Detect pdftex/xetex/luatex, and load appropriate font packages.
% This is inspired by the approach in the iftex package.
% pdftex:
\ifx\pdfmatch\undefined
\else
    \usepackage[T1]{fontenc}
    \usepackage[utf8]{inputenc}
\fi
% xetex:
\ifx\XeTeXinterchartoks\undefined
\else
    \usepackage{fontspec}
    \defaultfontfeatures{Ligatures=TeX}
\fi
% luatex:
\ifx\directlua\undefined
\else
    \usepackage{fontspec}
\fi
% End engine-specific settings

\usepackage{amsmath,amssymb}
\usepackage{fullpage}
\usepackage{graphicx}
\usepackage[svgnames]{xcolor}
\usepackage{url}
\urlstyle{same}


\usepackage[makestderr]{pythontex}
%\usepackage[depythontex]{pythontex}  % option neede if you want to depythonize the tex source, then run 
% % depythontex3 -o outputFileName.tex thisFileName.tex
%\usepackage[gobble=auto]{pythontex}
% \restartpythontexsession{\thesection}  % don't do this if you want to access variable in the scope of the whole document!


\usepackage[framemethod=TikZ]{mdframed}

\newcommand{\pytex}{Python\TeX}
\renewcommand*{\thefootnote}{\fnsymbol{footnote}}

%%%%%%%%%%%%%%%%%%%%%%%%%%%%%%%%%%%%%%%%%%%%%%%%%%%%%%%%%%%%%%%%
%%%%%%%%%%%%%%%%%%%%%%%%%%%%%%%%%%%%%%%%%%%%%%%%%%%%%%%%%%%%%%%%
%%%%%%%%%%%%%%%%%%%%%%%%%%%%%%%%%%%%%%%%%%%%%%%%%   pythontex <-


%%%%%%%%%%%%%%%%%%%%%%%%%%%%%%%%%%%%%%%%%%%%%%%%%%%%%%%%%%%%%%%%
%%%%%%%%%%%%%%%%%%%%%%%%%%%%%%%%%%%%%%%%%%%%%%%%%%%%%%%%%%%%%%%%
%%%%%%%%%%%%%%%%%%%%%%%%%%%%%%%%%%%%%%%%%%%%%%%%% mbpythontex -> 


%%%%%%%%%%%%%%%%%%%%%%%%%%%%%%%%%%%%%%%%%%%%%%%%%%%%%%%%%%%%%%%%
% import mbpytex from the file or copy mbpytex.py into the file>
%%%%%%%%%%%%%%%%%%%%%%%%%%%%%%%%%%%%%%%%%%%%%%%%%%%%%%%%%%%%%%%%

%%%%%%%%%%%%%%%%%%%%%%%%%%% import mbpytex ->
\begin{pycode}
from mbpytex.mbpytex import *
\end{pycode}
%%%%%%%%%%%%%%%%%%%%%%%%%%% import mbpytex <-

%%%%%%%%%%%%%%%%%%%%%%%%%%%%% mbpytex.py copied into the file to make the file standalone
%\begin{pycode}
%# how to keep fraction in sympy output (not implemented yet):
%# https://stackoverflow.com/questions/52402817/how-to-keep-the-fraction-in-the-output-without-evaluating
%
%def round_expr(expr, num_digits):
%    # how to print sympy numbers rounded to n decimals:
%    # https://stackoverflow.com/questions/48491577/printing-the-output-rounded-to-3-decimals-in-sympy
%    
%    import sympy as sp
%    return expr.xreplace(
%            {n : round(n, num_digits) for n in expr.atoms(sp.Number)})
%
%def matrix(M, mtype='matrix', dec=5):
%    """Return the latex representation of sympy or numpy matrix"""
%    # types of matrices can be found here:
%    # https://www.math-linux.com/latex-26/faq/latex-faq/article/how-to-write-matrices-in-latex-matrix-pmatrix-bmatrix-vmatrix-vmatrix
%    
%    import sympy as sp
%    
%    S = sp.Matrix(M) # converting to sympy Matrix in case M is a numpy matrix
%    
%    S = round_expr(S, dec) # rounding to specific number of decimals
%        
%    latexStr =  sp.latex(S)
%
%    if mtype!='matrix':
%        latexStr = latexStr.replace('matrix',mtype)
%
%    latexStr = latexStr.replace('\\left[','')
%    latexStr = latexStr.replace('\\right]','')
%    
%    return latexStr
%
%
%def pmatrix(M, dec=5):
%    """Return the latex representation of sympy or numpy matrix"""
%    # types of matrices can be found here:
%    # https://www.math-linux.com/latex-26/faq/latex-faq/article/how-to-write-matrices-in-latex-matrix-pmatrix-bmatrix-vmatrix-vmatrix
%    
%    return matrix(M,mtype='pmatrix',dec=dec)
%    
%def bmatrix(M, dec=5):
%    """Return the latex representation of sympy or numpy matrix"""
%    # types of matrices can be found here:
%    # https://www.math-linux.com/latex-26/faq/latex-faq/article/how-to-write-matrices-in-latex-matrix-pmatrix-bmatrix-vmatrix-vmatrix
%    
%    return matrix(M,mtype='bmatrix',dec=dec)
%
%def m2l(M):
%    """Wrapper function with short name"""
%    return bmatrix(M)
%\end{pycode}
%%%%%%%%%%%%%%%%%%%%%%%%%%%%% mbpytex.py copied into the file



%%%%%%%%%%%%%%%%%%%%%%%%%%%%%%%%%%%%%%%%%%%%%%%%%%%%%%%%%%%%%%%%
% import mbpytex from the file or copy mbpytex.py into the file<
%%%%%%%%%%%%%%%%%%%%%%%%%%%%%%%%%%%%%%%%%%%%%%%%%%%%%%%%%%%%%%%%



% Definition of folder names for LaTeX is loaded from mbpytex.py
%[preamble]

\begin{pycode}
import  numpy as np
import  sympy as sp

# global project parameters
auxFilesFolder = 'auxFiles/'
figuresFolder  = 'figures/'
\end{pycode}

\newcommand{\auxFilesFolder}{auxFiles/}
\newcommand{\figuresFolder}{figures/}

% % Commands like this cannot be defined if you want to 
% % depythonize the document later.

%\newcommand{\mtx}[1]{\py{matrix(#1)}}
%\newcommand{\mtxb}[1]{\py{bmatrix(#1)}}
%\newcommand{\mtxp}[1]{\py{pmatrix(#1)}}
%%%%%%%%%%%%%%%%%%%%%%%%%%%%%%%%%%%%%%%%%%%%%%%%%%%%%%%%%%%%%%%%
%%%%%%%%%%%%%%%%%%%%%%%%%%%%%%%%%%%%%%%%%%%%%%%%%%%%%%%%%%%%%%%%
%%%%%%%%%%%%%%%%%%%%%%%%%%%%%%%%%%%%%%%%%%%%%%%%% mbpythontex <-

%\usepackage[polish]{babel}
\usepackage[english]{babel}

\usepackage[utf8]{inputenc}
\usepackage[T1]{fontenc}
%\usepackage[a4paper,left=3.5cm,right=2cm,top=2cm,bottom=2.5cm,includefoot=false,includehead=false,footskip=1.16cm]{geometry} 	% page dimensions




\usepackage{subfigure}

% % % % % % % % % % % % % COMMENTS 
% % short or inline comments, or TODO comments
% if you want to compress all
\usepackage[colorinlistoftodos,prependcaption,textsize=tiny]{todonotes}
%\usepackage[colorinlistoftodos,prependcaption,textsize=tiny,disable]{todonotes}

\usepackage{xargs}    % Use more than one optional parameter in the new commands
\usepackage{soulutf8}          % \ul for diacritics - used for striking out in \MBe command

% margin note
\newcommandx{\MB}[2][1=]{\todo[linecolor=red,backgroundcolor=red!25,bordercolor=red,#1]{#2}}
% plain text note 
\newcommand{\MBp}[1]{{\leavevmode\color{red!75}#1}}
% edit note
\newcommand{\MBe}[2]{{\leavevmode\color{red!75}{#1}}{{\leavevmode\color{red!50}\st{#2}}}}

% % longer parts of the comments 
\usepackage{comment}
\includecomment{private-comment}
% \excludecomment{private-comment}
\includecomment{answers}
%\excludecomment{answers}


\newtheorem{definition}{Definicja}
\newtheorem{exercise}{Zadanie }
\newtheorem{solution}{Rozwiązanie }
\newtheorem{theorem}{Twierdzenie }
\newtheorem{task}{Zadanie do samodzielnego wykonania }
\newtheorem{remark}{Uwaga }
\newtheorem{proof}{Dowód }

%% dots after section/subsection number in Table of Contents
%\usepackage{tocloft} 					
%\renewcommand{\cftsecaftersnum}{.}
%\renewcommand{\cftsubsecaftersnum}{.}
%\renewcommand{\cftsubsubsecaftersnum}{.}



\title{\pytex\ stencil and examples}
\author{MB}



\begin{document}

\maketitle

\begin{abstract}
This document is a part of a stencil of a reproducible research approach project with a small additional auxiliary python package for easy handling numpy/sympy matrices printing in ...
\end{abstract}


\section{Introduction}

\section{Examples of pythontex commands, their usage and/or description}

\begin{itemize}

\item 
\verb|begin{pycode}| -- executes the code inside the environment. If any \verb|print| function is run inside it, the results of printing are printed 
-- code is executed and not typeset

Example:

\begin{verbatim}
# This is not executable part of the document! It's only a text in verbatim!
\begin{pycode}
a = 123.45
s = 'This is string'
b = a+2

print(b)
print(s)
\end{pycode}
\end{verbatim}

produces:

\begin{pycode}
a = 123.45
s = 'This is string'
b = a+2

print(b)
print(s)
\end{pycode}


\item 
\verb|begin{pysub}| -- " useful for inserting Python-generated content in contexts where the normal \verb|\py| and \verb|\pyc| would not function or would be inconvenient due to the restrictions imposed by LATEX. Since Python processes $\langle code \rangle$ and performs substitutions before the result is passed to LATEX, substitution fields may be anywhere."

{\bf I stumbled upon the errors when depythonizing the source while having \verb|\pysub| environment in it}. That is why \verb|\pysub| code is commented out in this document.

%\begin{pysub}
%a = 123.45
%s = 'This is string'
%b = a+2
%
%!{print('AAAAAAAAA')}
%
%print(b)
%print(s)
%\end{pysub}

\item 
\verb|begin{pyverbatim}| -- like \verb|begin{verbatim} + begin{pycode}| -- code is typeset and not executed

\item 
\verb|begin{pyblock}| -- the same as \verb|begin{pycode}| but the code inside the environment is additionally printed 
-- code is typeset and executed

\item Examples with \verb|\py, \pyc, \pyb| 

\begin{itemize}
\item \verb|\py| -- prints out the value of the variable or the command, provided it can be stringified

\py{a}

\py{10*2+4}

\py{s}

\item \verb|\pyc| -- inline version of \verb|begin{pycode}|

-- execute, not typeset

\pyc{1+1}

\pyc{print(1+1)}

\pyc{print(a)}


\item \verb|\pys| -- prints out what is inside of the braces (kind of verbatim environment) -- like \verb|\begin{pysub}|

\pys{1+1}

\pys{a}

\pys{print(a)}

\item \verb|\pyb| -- like \verb|\pys| but it colourizes the printout

\pyb{1+1}

\pyb{a}

\pyb{print(a)}

\end{itemize}

\end{itemize}

\section{Examples of usage of mbpytex auxiliary functions}


\subsection{Pretty printouts of numpy/sympy matrices}

\subsubsection{numpy}
\begin{pyblock}
import numpy as np

A = np.array([[1,2,1],
              [0,3,1],
              [2,1,1]])
              
v = np.array([[2],
              [0],
              [1]])
              
b = A@v
\end{pyblock}

Since the following lines 
\begin{verbatim}
\begin{pycode}
from mbpytex.mbpytex import *
\end{pycode}
\end{verbatim}
are included at the beginning of the document, now we can write

\begin{verbatim}
\begin{equation}
\left[
\py{matrix(A)}
\right]
\end{equation}
\end{verbatim}
%%%%or
%%%%\begin{verbatim}
%%%%$\py{matrix(A)}$
%%%%\end{verbatim}

which produces

\begin{equation}
\left[
\py{matrix(A)}
\right]
\end{equation}

%%%%$\py{matrix(A)}$

\vspace{0.5cm}

We could even latexify pythontex commands like this! (see the beginning of the file) and hide pythontex under the latex commands:
\begin{verbatim}
\newcommand{\mtx}[1]{\py{matrix(#1)}}
\newcommand{\mtxb}[1]{\py{bmatrix(#1)}}
\newcommand{\mtxp}[1]{\py{pmatrix(#1)}}
\end{verbatim}

Then we could use \verb|$\mtxb{A}$| to typeset the matrix.
% $\mtxb{A}$
{\bf But } there are problems with depythonizing documents with commands like the ones above. So, more versatile solution would be to avoid using those commands.

%% % % % % % % % % % % % % % % % % % % % % % % % % % % % %

\subsubsection{sympy}

We can do sympy calculations in plain \verb|\pyblock| after importing numpy:
\begin{pyblock}
from sympy import *

x = symbols('aa')
k = Symbol("k")
As1 = Matrix([[k,2,1],
              [0,k,1],
              [2,1,k]])

ss = latex(As1)
\end{pyblock}

and access the variables in the same way as above: $\py{ss}$

But we can also use special environments with \verb|sym| prefix added, like \verb|sympyblock|. There are two remarks that need to be made here:
\begin{itemize}
\item The difference in usage between those two approaches is that in \verb|sympyblock| there is no need to import sympy. It is imported by default like if we used ''\verb|from sympy import *|'' in \verb|pyblock| environment.

\item Commands \verb|sympyblock| and \verb|pyblock| are run in {\bf two different sessions}. So the {\bf variables from the one context are not accessible from inside the other one}.
\end{itemize}


\begin{sympyblock}
# no need to import sympy. It is already done by sympyblock environement!

k = Symbol("k")

As2 = Matrix([[1,2,1],
              [0,k,1],
              [2,1,1]])
              
vs2 = Matrix([[2],
              [0],
              [1]])

bs2 = As2@vs2
bs2 = As2 @ vs2
\end{sympyblock}

Now, with \verb|\sympy{latex(As2)}| we can obtain:
$\sympy{latex(As2)}$.

Remember, that now you cannot do:
\verb|\sympy{latex(As1)}| or \verb|\py{latex(As2)}| because \verb|As1| was created in the context of \verb|\py| environment, and \verb|As2| was created in the context of \verb|\sympy| environment. 

Since we want to have homogeneous access both to numpy and to sympy libraries, in \verb|mbpytex| we assume that the main document context is \verb|\py|/,\verb|\pyblock|/\verb|\pycode|, etc... and we need to import both numpy and sympy libraries explicitly in it.

\section{Summary - cheat sheet of mbpytex library}

\subsection{Matrix typesetting}
Use: \verb|\py{m2l(matrixname)}| command or its less concise versions \verb|\py{matrix(matrixname)}|, \verb|\py{matrixb(matrixname)}|, \verb|\py{matrixp(matrixname)}|..., regardless whether you have numpy or sympy matrix.

$\py{m2l(np.array([[1,2,1],
               [0,3,1],
               [2,1,1]]))}$
$\py{m2l(sp.Matrix([[1,2,3],
                 [2,3,4],
                 [3,4,5]]))}$

\begin{pycode}
A = np.array([[1,2],
               [0,3]])

x = np.array([[1],
               [0]])

b = A @ x
S = sp.Matrix([[1,2],
               [0,3]])
\end{pycode}

m2l = $\py{m2l(A)}$, 
\quad
wektor b = $\py{m2l(b)}$
\quad 
\subsection{Plots with matplotlib}

We can create plots with matplotlib, perfectly matching the plot fonts with the document fonts.  No more searching for the code that created a figure!

It is possible to pass page dimensions and similar contextual information from the \LaTeX\ side to the Python side.  If you want your figures to be, for example, a particular fraction of the page width, you can pass the value of \pygment{latex}{\textwidth} to the Python side, and use it in creating your figures.  See \pygment{latex}{\setpythontexcontext} in the main documentation for details.

You may want to use matplotlib's PGF backend when creating plots.

\begin{pyblock}
import matplotlib.pyplot as plt
import numpy as np

plt.rc('text', usetex=True)
plt.rc('font', family='serif')
plt.rc('font', size=10.0)
plt.rc('legend', fontsize=10.0)
plt.rc('font', weight='normal')

x = np.linspace(0, 10)
plt.figure(figsize=(4, 2.5))
plt.plot(x, np.sin(x), label='$\sin(x)$')
plt.xlabel(r'$x\mathrm{-axis}$')
plt.ylabel(r'$y\mathrm{-axis}$')
plt.legend(loc='lower right')
plt.savefig(figuresFolder+'myplot.pdf', bbox_inches='tight')
plt.savefig(figuresFolder+'myplot.png', bbox_inches='tight')
\end{pyblock}

\begin{center}
\includegraphics{\figuresFolder myplot.pdf}
\end{center}



\subsection{Using comments}

\subsubsection{'todo' package}
For short, inline comments or TODO comments use \verb|todo| package like this:
\MB{This is todo margin note}

\MBp{This is short inline comment}

%\MBe{This is old part of text}{Proposal of the new version}
\MBe{New}{Old}

\subsubsection{'comment' package}
For larger parts of the text use \verb|comment| environment. After defining the specific environments at the beginning of the document we can use them like this:

\vspace{0.5cm}

\begin{private-comment}
This is a text with a private comment which will be excluded in the final, official version of the manuscript.
\end{private-comment}

\vspace{0.5cm}

\begin{exercise}\
This is a text of an exercise that will be visible in the final version of the manuscript.

\end{exercise}

\begin{answers}
This a test of an answer that will be excluded from the final version of the manuscript.
\end{answers}



\end{document}
